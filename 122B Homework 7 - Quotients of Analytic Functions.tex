\documentclass[11pt]{article}
\usepackage[margin=1.25in]{geometry}

\usepackage{graphicx,tikz}
\usepackage{amsmath,amsthm}
\usepackage{amsfonts}
\usepackage{amssymb}
\usepackage{boondox-cal}
\title{ }

\newtheorem*{thm}{Theorem}

\begin{document}
	%\maketitle
	%\date
	\begin{center}	% centers
		\Large{Homework 7}	% Large makes the font larger, put title inside { }
	\end{center}
	\begin{center}
		Vincent La \\
		Math 122B \\
		August 29, 2017
	\end{center}
	
\begin{enumerate}
	\item Show that
	\[Res_{z=\pi i} \frac{\exp{zt}}{\sinh{z}} + Res_{z=-\pi i} \frac{\exp{zt}}{\sinh{z}} = -2\cos{\pi t} \]
	
	\begin{proof}
		Let us tackle this problem by separating evaluating the numerators and denominators.
		
		\bigskip
		
		First, let us evaluate the denominators. In general, because
		$\sinh{z} = -i\sin{iz}$
		it follows that 
		$\frac{d}{dz} \sinh{z} = -i^2 \cdot \cos{iz} = \cos{iz}$
		Therefore, the denominators are
			\[\begin{aligned}
			\cos{i(\pi i)} &= \cos{\pi i^2} = \cos{-\pi}\\
			\cos{i(-\pi i)} &= \cos{-\pi i^2} = \cos{\pi}\\			
			\end{aligned}\]
			
		Both of these expressions are -1, so our residues share a common denominator.
		
		\bigskip
		
		Then, recall the identity
		$\exp{z} = \exp{x}\cos{y} + i \exp{x}\sin{y}$
		where $x$ and $y$ are the real and imaginary parts of $z$ respectively. Therefore,
		\[\begin{aligned}
		\exp{\pi it}
			&= \exp{0}\cos{\pi t} + i\exp{0} \sin{\pi t} \\
			&= \cos{\pi t} + i\sin{\pi t} \\
		\exp{-\pi it}
			&= \exp{0}\cos{-\pi t} + i\exp{0} \sin{-\pi t} \\		
			&= \cos{-\pi t} + i\sin{-\pi t}
		\end{aligned}\]
		
		\bigskip
		
		Simplifying the original expression, we thus get
		\[ Res_{z=\pi i} \frac{\exp{zt}}{\sinh{z}} + Res_{z=-\pi i} \frac{\exp{zt}}{\sinh{z}}
		=  \frac{ (\cos{\pi t} + \cos{-\pi t}) + i( \sin{\pi t} + \sin{-\pi t} )}{
				   -1} \]
				
		Because $\cos$ is an even function, and $\sin$ is an odd function, this simplifies to $-2\cos{\pi t}$ as we set out to prove.
	\end{proof}
			
	\item Evaluate the integral
	\[ \int_C \frac{dz}{\sinh{2z}} \]
	about $C: |z| = 2$.
	
	\paragraph{Solution} First, notice that 
	\[\sinh{2z} = i \sin{i 2z} \]
	is differentiable everywhere except for the singularities that occur whenever $\sin{i 2z} = 0$. These occur when $z = \frac{i \pi n}{2}$ for $n \in \mathbb{Z}$. Within the circle $|z| = 2$, this gives us the isolated singular points
	$z = 0, \pm \pi i, \pm 2 \pi i$. Now, applying the chain rule we get $\frac{d}{dz} \sinh{2z} = 2\cos{i 2z}$. Therefore, our residues are
	\[\frac{1}{ \cos{0} } + \frac{1}{ \cos{\pm 2\pi}} + \frac{1}{ \cos{\pm 4\pi}} \]
	
	Now, recall that $\cos(0)$ is an even function with period $2\pi$, so the above is really just,
	\[5 \cdot \frac{1}{ \cos{0} } = 5 \]
	
	Applying Cauchy's Residue Theorem,
	\[\int_C \frac{dz}{\sinh{2z}} = 2\pi i \cdot \sum Res = 2\pi i \cdot 5 = 
	  10\pi i\]
\end{enumerate}
\end{document}