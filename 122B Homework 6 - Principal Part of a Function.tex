\documentclass[11pt]{article}
\usepackage[margin=1.25in]{geometry}

\usepackage{graphicx,tikz}
\usepackage{amsmath,amsthm}
\usepackage{amsfonts}
\usepackage{amssymb}
\usepackage{boondox-cal}
\title{ }

\newtheorem*{thm}{Theorem}

\begin{document}
	%\maketitle
	%\date
	\begin{center}	% centers
		\Large{Homework 6}	% Large makes the font larger, put title inside { }
	\end{center}
	\begin{center}
		Vincent La \\
		Math 122B \\
		August 22, 2017
	\end{center}
	
\section{Homework}
\begin{enumerate}
	\item Write the principal part of the following functions at the singular point and determine whether that point is a pole, a removable singular point, or an essential singular point:
		\begin{enumerate}
			\item $\frac{z^2}{1 + z}$
			\paragraph{Solution}
			
			This function has a singular point at $z = -1$.
			First, we will find the Laurent series expansion. Notice that,
			\[\begin{aligned}
			z^2
			&= [(z + 1)^2 - 2z - 1] \\
			&= [(z + 1)^2 - 2(z + 1) + 1] \\
			\end{aligned}\]
			
			Therefore,
			\[
			\begin{aligned}
			\frac{z^2}{1 + z}
			&= z^2 \cdot \frac{1}{1 + z} \\
			&= [(z + 1)^2 - 2(z + 1) + 1] \cdot \frac{1}{1 + z} \\
			&= \frac{(z + 1)^2 - 2(z + 1) + 1}{1 + z} \\
			&= (z + 1) - 2 + \frac{1}{1 + z}
			\end{aligned} \]
			
			By inspection, the principal part of this function is 
			\[\frac{1}{1 + z}\]
			so $z = -1$ is a pole of order 1.			
			
			\item $\frac{\sin{z}}{z}$
			\paragraph{Solution}
			First, we will try to find a Laurent series representation for $f(z)$.
			
			\[\begin{aligned}
			\frac{\sin{z}}{z}
			&= \frac{1}{z} \cdot \sin{z} \\
			&= \frac{1}{z} \cdot \sum^{\infty}_{n=0} (-1)^n \frac{z^{2n + 1}}{(2n + 1!)}
			& (|z| < \infty) \\
			&= \frac{1}{z} \cdot [(-1)^0 \cdot \frac{z^1}{1!} +
				(-1)^1 \frac{z^3}{3!} + (-1)^2\frac{z^5}{5!} + ...] & \text{Expanding the series} \\
			\end{aligned}\]
			
			By inspection, this Laurent Series has no terms with negative exponents and is really just a regular power series. Therefore, it is a removable singular point.			
		\end{enumerate}
		
	\item Show that the singular point of each function is a pole and determine its order $m$ and find the corresponding residue:
		\begin{enumerate}
			\item $(\frac{z}{2z + 1})^3$
			\paragraph{Solution}
			First, this function has singularities whenever $(2z + 1)^3 = 0$, implying $z = -\frac{1}{2}$ is a singular point.
			
			Now, notice that
			\[\begin{aligned}
			\frac{z}{2z + 1}
			&= \frac{z}{2} \cdot \frac{1}{z + \frac{1}{2}} \\
			\end{aligned}\]
			
			Furthermore, because $z = (z + \frac{1}{2}) - \frac{1}{2}$, the above is equivalent to
			\[\begin{aligned}
			\frac{z}{2z + 1}
			&= \frac{(z + \frac{1}{2}) - \frac{1}{2}}{2} \cdot
				\frac{1}{z + \frac{1}{2}} \\
			&= \frac{1}{2} \cdot \frac{
				(z + \frac{1}{2}) - \frac{1}{2}}{
				z + \frac{1}{2}} \\
			&= \frac{1}{2} \cdot [1 - \frac{0.5}{z + \frac{1}{2}}] \\
			\end{aligned}\]
			
			Finally, we get
			\[\frac{1}{2} - \frac{0.25}{z + \frac{1}{2}}\]
			
			Therefore, because there is only $(z - z_0)^n$ term with an exponent of $-1$, this is a pole of order 1 with residue 0.25.
			
			\item $\frac{\exp{z}}{z^2 + \pi^2}$
			\paragraph{Solution}
			Consider $z = \pi i$. At this point, the denominator $z^2 + \pi^2 = \pi^2 \cdot i^2 + \pi^2 = 0$, so $\pi i$ is a singular point.
			
			\bigskip
			
			First, let us find the Taylor Series of $\exp(z)$ about $z = \pi i$. Because
			$\frac{d}{dz} \exp(z) = \exp(z)$, and $\exp(\pi i) = -1$,
			\[\begin{aligned}
			\frac{f^{(n)}(z_0)}{n!}
			&= \frac{\exp(\pi i)}{n!} \\
			&= \frac{-1}{n!} \\
			\end{aligned}\]
			
			Thus, the Taylor Series of $\exp(z)$ centered at $\pi i$ is 
			\[ f(z) = \sum^{\infty}_{n=0} \frac{-1}{n!} (z - \pi i)^n \]
			which converges for all $z \in \mathbb{C}$.
			
			Notice that 
			\[(z + \pi i)(z - \pi i)
			= (z^2 - z\pi i + z\pi i - \pi^2 i^2)
			= z^2 + \pi^2 \]
			
			Therefore,
			\[\begin{aligned}
			\frac{\exp{z}}{z^2 + \pi^2}
			&= \frac{1}{(z + \pi i)(z - \pi i)} \cdot 
				\sum^{\infty}_{n=0} \frac{-1}{n!} (z - \pi i)^n \\
			&= \frac{1}{z + \pi i} \cdot 
				\sum^{\infty}_{n=0} \frac{-(z - \pi i)^{n - 1}}{n!} \\
		    \end{aligned}\]
		    
		    And the completion of this solution has been left as an exercise for the reader.
		\end{enumerate}
\end{enumerate}
\end{document}