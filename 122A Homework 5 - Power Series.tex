\documentclass[11pt]{article}
\usepackage[margin=1.25in]{geometry}

\usepackage{graphicx,tikz}
\usepackage{amsmath,amsthm}
\usepackage{amsfonts}
\usepackage{amssymb}
\usepackage{boondox-cal}
\title{ }

\newtheorem*{thm}{Theorem}

\begin{document}
	%\maketitle
	%\date
	\begin{center}	% centers
		\Large{Homework 5}	% Large makes the font larger, put title inside { }
	\end{center}
	\begin{center}
		Vincent La \\
		Math 122A \\
		July 23, 2017
	\end{center}
	
	\section{Monday and Tuesday}
	\begin{enumerate}
		\item[4.3] Evaluate $\int_C f$ where $f(z) = x^2 + iy^2$ as in Example 1, but where $C$ is given by $z(t) = t^2 + it^2$, $0 \leq t \leq 1$.
		
		\paragraph{Solution} This implies $\dot{z}(t) = \cos(t) - i\sin(t)$, so
		\[\int^b_a f(z(t))\dot{z}(t) dt = \int^{2\pi}_0 (\frac{\sin(t)}{\sin^2(t) + \cos^2(t)} - i\frac{\cos(t)}{\sin^2(t)+\cos^2(t)})(\cos(t) - i\sin(t) dt\]
		
		This in turn simplifies to
		
		\[\begin{aligned}
		\int^{2\pi}_{0} (\sin(t) - i\cos(t))(\cos(t) - i\sin(t)) dt &=
		\int^{2\pi}_{0} \sin^2(t) + \cos^2(t) dt \\
		&= -\int^{2\pi}_{0} 1 dt \\
		&= -2\pi \\
		\end{aligned}\]
		
		This is unlike Example 2, where the answer was $2\pi i$. Now, line integrals should be independent of parameterization, but this parameterization is actually the same as the one in Example 2 but moving in the opposite direction.
		
		\item[4.5] Use the Fundamental Theorem of Calculus (Proposition 4.12) to prove that if $F$ is analytic on a region and $F'(z) = 0$ then $F$ is constant.
		
		\begin{proof}
			Let $F$ be any analytic function where $F'(z) = 0$. Let the bounds of integration $a, b \in \mathbb{Z}$ be arbitrary. First, the Fundamental Theorem of Calculus states that
			\[0 = F'(z) = \int_C f(z) dz = F(z(b))] - F(z(a)) \]
			
			Thus,
			\[0 = \int 0 dz = F(z(b)) - F(z(a)) \]
			
			This is only true if $a = b$ or $F$ is constant. However, since $a, b$ arbitrary, it must be that $F$ is constant.
		\end{proof}
	\end{enumerate}
	
	\section{Wednesday and Thursday}
	\begin{enumerate}
		\item Prove that $\lim_{n \rightarrow \infty} \xi^n = 0$ whenever $|\xi| < 1$. Conversely, prove that if $|\xi| \geq 1$ then $\{\xi^n\}$ is a divergent sequence.
		
		\begin{enumerate}
			\item $\lim_{n \rightarrow \infty} \xi^n = 0$ whenever $|\xi| < 1$
			
			\begin{proof}
				Let $\epsilon > 0$ be arbitrary and let $\delta = \frac{1}{\epsilon}.$ Then, show that whenever $|n| > \delta$ that $|f(n) - 0| < \epsilon$. Notice,
				$n > \delta = \frac{1}{\epsilon}$ implies that
				\[|\xi^n| < |\xi^\frac{1}{\epsilon}| < 1\]
				because $|\xi| < 1$.
				
				\bigskip
				
				This further implies that $|\xi^\frac{1}{\epsilon}|^\epsilon < \epsilon$, or
				$|\xi^n| < |\xi^1| < \epsilon$ as we set out to prove.
			\end{proof}
			
			\item
			
		\end{enumerate}
	\end{enumerate}
	
\subsection{Extra Problems}
\begin{enumerate}
	\item[5.1] Find the powers series expansion of $f(z) = z^2$ around $z = 2$.
	
	In general, a Taylor Series about a function is 
	\[\sum^{\infty}_{n=0} \frac{f^{(n)}(a)}{n!}(z-a)^n \]
	
	Filling the terms in, here it is $4 + 4(z-2) + (z-2)^2$.
	
	\item[5.2] Find power series expansion for $e^z$ about any point $a$
	
	\item[5.3] Show that an odd entire function only has odd terms in its power series expansion about $z = 0$. 
	
	\begin{proof}
		(Insert proof that $f$ even implies $f'$ odd and that $f$ odd implies $f'$ even here).
		
		If an odd function is entire, it has a power series expansion which we can write
		\[
		\frac{f^{(0)}(0)}{0!}z^0 +
		\frac{f^{(1)}(0)}{1!}z^1 +
		\frac{f^{(2)}(0)}{2!}z^2 + 
		\frac{f^{(3)}(0)}{3!}z^3 + ... \]
		
		Now, suppose that if the $n^{th}$ term is odd, then the $n + 1^{th}$ term will also be odd and induct on $n \in \mathbb{Z}^+$.
		
		First, let us show the base case $n = 0$. Now, because $f$ is odd, then $\frac{f^{(0)}(0)}{0!}z^0 = f(0)$ is odd.
	\end{proof}
\end{enumerate}
\end{document}